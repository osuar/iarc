\documentclass[12pt,letterpaper]{article}
\usepackage{graphicx}

\begin{document}

\begin{titlepage}
    \vspace*{4cm}
    \begin{center}
    {\huge
        OSURC Aerial team\\[1cm]
    }
    {\large
	IARC 2012 Paper\\
    }

    \end{center}
    \vfill
\end{titlepage}
\section{abstract}
The Oregon State University Aerial Robotics Team has the goal to develop
an indoor autonomous flight platform to compete in the International Aerial
Robotics Competition. Our team is comprised of undergraduate students in the
Mechanical, Industrial, and Manufacturing Engineering and Electrical Engineering
and Computer Science departments. Over the past year we have developed a quad
rotor flight platform along with a Robot Operating System (ROS) base station
interface in order to navigate indoor environments. A robotic hand is also
being developed in order to handle manipulating the environment (picking up the
flash drive).

\section{Introduction}
	The Oregon State University strategy for indoor automous flight involves
specific functional goals for each part of the system. Flight stability is
handled by the flight platform, however, all navigational data gathered from
distance sensors and cameras mounted on the flight platform is transmitted to
the base station for processing. The base station then transmits navigational
commands to the flight platform.
	Stabilization sensory is handled using a three axis gyroscope, 
accellerometer and magnetometer. Navigational data comes from a number of
distance sensors mounted on servos. This solution was chosen because of the
low electrical and computing power required to process distance sensor data.
A seperate wireless camera is also implemented to provide object recognition
through the OpenCV libraries.

\section{Electrical}
The electrical setup for our quad rotor runs primarily off of a 3 cell
lithium polymer battery. The ~12V from they battery is used to power the motors
and motor controllers on the flight platform. This supply is also regulated down
to 5V to power an Atmel Xmega microcontroller used as the central intelligence
on the flight platform, an xbee radio for wireless communication, and 
addionally to power the various distance and inertial sensors. A wireless camera
is powered by a seperate 9V source.

\section{Flight Control}
The on board flight control system was developed to provide basic
stability for the flight platform. It does not have contingencies to prevent
drift. Integration of gyroscope samples is used to create an estimate of the
orientation of the flight platform. This combined with the raw data from the
gyroscopes is as input to a position differential (PD) controller. To prevent
the gyroscope positional estimates from drifting they are averaged with 
magnetometer and accelerometer samples.

\section{Flash Drive Pickup}
The IARC competition requires the use of a grasping mechanism for one of the 
primary objectives.  The chosen design for the grasping mechanism will make use 
of an under-actuated, passively compliant four-fingered hand.  Each of the four 
fingers will contain two flexure joints, rather than rotational bearing joints. 
A fully actuated hand would require eight actuators; this under-actuated design 
will require only one.  A single actuator will control all joints through the 
use of a pulley system, allowing each finger link to actuate until it comes into
 contact with an object.  All tendon cables will see the same force from the 
actuator.  Not only does this design greatly reduce cost, but it saves 
weight--an essential benefit for an aerial vehicle.  In addition, the design all
ows the hand to automatically adapt to the shape of any object, without the need
 for special positioning and calculated movements.  

\section{Chasis}
The new chassis design will make use of lighter weight carbon fiber tubing.  The
 current design consists of 4 motors mounted on arms that mount to a central 
chassis.  The new design will consist of only two solid arms that mount to one 
another, each connecting a pair of motors.  This will remove bending moments 
resulting from motor forces from the central chassis, allowing for material 
reduction and weight savings.  The current chassis also has a tendency to snag 
the ground when the aerial vehicle lands with significant lateral speed.  This 
causes the vehicle to tip over, sending its propellers into the ground and 
causing damage.  The new chassis will have less skid resistance, while allowing 
for vertical impact absorption.  Blade guards will also be added around the 
perimeter of the propellers, to reduce the risk of propeller damage from 
incidental contact with the environment.

\end{document}
